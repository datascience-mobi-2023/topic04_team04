\section{Introduction}\label{sec:introduction}

The Dengue virus is a vector borne virus, consisting of four serotypes (DENV 1-4).  It is transmitted to humans by mosquitoes, more specifically by \emph{Aedes aegypti} and \emph{Aedes albopictus} \citep{Phanitchat.2019}. The resulting infection goes asymptomatic in many cases, but can also cause Dengue Hemmorhagic Fever (DHF), characterized by symptoms like fever, headache, joint and muscle pain as well as (internal) bleeding and bruising \citep{Gubler.1998}.
Dengue is an emerging public health issue, with half of the worlds population at risk of infection. 390 million cases per year are estimated worldwide, with most of them not showing symptoms and thus not being reported. Affected areas range from sub-tropical to tropical regions, with south-east Asia, including Thailand being one of the most seriously affected regions. Newly affected areas also include Europe \citep{WHO.2023}.
The disease control of dengue is challenging due to the absence of an effective treatment or vaccine, which leaves only the treatment of symptoms with painkillers like paracetamol \citep{WHO.2023}. Responsible institutions in affected areas currently focus on prevention, vector control, case control and prediction of possible future outbreaks \citep{Phanitchat.2019}. 

An important factor in predicting the epidemiological dynamics of Dengue is the climate: climate fluctuations due to recurring weather phenomenons and climate change are shown to influence \emph{Aedes} biology and infections \citep{Descloux2012, Phanitchat.2019}.
In several studies, maximal temperature has an effect on dengue transmission \citep{Descloux2012}, being associated with higher incidence \citep{Phanitchat.2019} between 27 °C and 29,5 °C. Extreme global climate events like el Niño have been shown to affect disease outbreaks like Dengue as well \citep{Anyamba2019}.

In Thailand, significant climate factors are monsoon (hot and rainy) between may and october and dry season between november and april. It has been shown that in northern districts of Thailand, there has been a detectable rise in temperature since the mid 20th century \citep{Masud2016}.

In this analysis, we are going to investigate the correlation between temperature and Dengue in Thailand from 2006 to 2020 to assess the significance of climate change, recurring climate fluctuations and extremes and geographical factors on Dengue infections. This will allow us to model a forecast for the development of Dengue infections in Thailand in the near future. 
