\section{Discussion}
Our results show that increasing temperatures, resulting from climate change as well as recurrent weather phenomenons, are significantly associated with increasing DHF cases.
The relationship between temperature and DHF did not show a significant correlation, but a non-linear pattern was identified. DHF cases exhibited an increase with temperature up to 28°C, followed by a subsequent decrease. Similar temporal patterns were observed, indicating that high temperatures align with high case numbers. By comparing the spatial distribution of temperature and DHF, certain areas with similar patterns were identified. However, it cannot be concluded that provinces experiencing high temperatures also exhibit high case numbers. 
The characterized relationship could be observed in predictions as well. 

By a linear model, a general temperature increase over the studied time period can be shown. Nevertheless, it has to be taken into account that the temperature development over the years is not a linear process due to the many influencing factors. Thus, linear models can not be used for exact predictions or identification of patterns. 

It was shown that DHF cases and temperature have a similiar seasonal pattern and general trend: Time periods with high temperatures correspond to periods with high case numbers. Thus, there is indeed a strong relationship between temperature and dengue cases. However, whether there is a causal relationship cannot be concluded from Figures \ref{fig:decomp_ts_dengue} and \ref{fig:Trend_temp_cases}. Nevertheless, the observation that rising temperatures preceded the rise of dengue cases by a few months suggests an influence of high temperatures on \emph{Aedes} biology and virus transmission. 
Higher temperature can indeed positively influence Dengue transmission, as it alters \emph{Aedes} biology: It makes reproduction of the mosquitoes more efficient by shortening the maturation period of  larvae. Virus transmission is promoted as well, as the virus needs less time to spread in its host and make it infectious. These two effects lead to increasing numbers of infectious mosquitoes \citep{Anyamba2019}.
 
The trends in Figure \ref{fig:Trend_temp_cases} show the second highest DHF case numbers in 2015-16, which is the period of one of the strongest El Niño events since 1950. El Niño increased the temperature in Thailand in that timeframe, leading to higher DHF cases due to effects of temperature on Dengue infections elaborated above \citep{Anyamba2019}, which our analysis confirms. Previous studies have also found this effect by applying an autoregressive model: Similar effects of El Niño on Dengue incidences occured between 1996 and 2005 \citep{Tipayamongkholgul2009}.
Another El Niño event took place in 2018-19, which was another time period with high DHF cases.  In 2019, El Niño led to early monsoon onset \citep{Hu2020}, a possible contributing factor to Dengue transmission. 

 The optimal temperature has been observed to lie at 29.3 °C with an small daily range of deviation \citep{Liu2014}. Another study also proved 29°C to be the temperature of maximal dengue transmission \citep{Phanitchat.2019}. They found 80\% of dengue cases occurring at a mean temperature of between 27.0 and 29.5 °C. Our analysis proved 61\% of infections to occur between 27 and 29.5 °C, aligning with both sources (see Figure \ref{fig:smooth_spline}).
 
In the maps showing mean DHF cases in the different provinces, the northwestern provinces have comparatively high incidences, although the temperatures are lower (see Figure \ref{fig: }). A possible cause for this unexpected observation could be frequent floods during south-west monsoon, associated with high incidences in that region (e.g. Mae Hong Son) \citep{Chaithong2022}. The connection of floods and incidences could result from increased water sources for mosquito breeding grounds and loss of preventive actions and healthcare resources. 

To evaluate the further development of DHF cases in Thailand in the future, two different models were applied for forecasting. The ARIMA model projected a seasonal pattern with a slight upward trend in Thailand from 2021 to 2024, capturing the fundamental characteristics of case development in the country. However, a comparison of the model's predictions for 2017 to 2020 with the reported cases during that period revealed discrepancies in the amplitudes of the model. Since the model only relies on past cases and does not consider nonlinear trends, it cannot accurately predict complex relationships influenced by various factors. Nevertheless, the actual cases still lie in the deviation range, proving the general validity of the model.

Secondly, the GAM was used to predict future DHF cases based on the temperature. The edf value is with 13.82 very close to k with 16, which indicates that the model is using most of the available flexibility to capture the underlying relationship. The edf value is far from 1, therefore a linear correlation can be eliminated. It suggests a wigglier spline as well. It was shown that the smooth function of the temperature has a highly significant impact on the predicting incidence, but the model only explains 4.7\% of the variance in the data. Thus, it can be said that temperature is one significant of many parameters which will influence the development of dengue.
In the future, the east of Thailand will be the main origin of dengue fever outbreaks. This could be very critical, because of the low coverage of health care establishments \citep{Witthayapipopsakul2019}. Although the east will be a hotspot, the model suggests that the average incidence in Thailand will increase compared to the period of 2006 to 2020. 

High fluctuation in the incidence values (sd = 15.7) can’t be seen in temperature values (sd = 2.1 °C). The difference in the standard deviation is even higher in the south monsoon season with a sd for the incidence of 21,6 and a sd for the temperature of 1.2 °C. At this time of the year, other parameters like precipitation have a higher influence on dengue infection. In south east asia the temperature doesn’t fluctuate much over the year (Quelle).
Regional patterns are probably more outbreak specific and may be influenced by extreme weather events. The sd of the average temperature per region is 1.1 °C, which indicates that temperature values are relatively stable over Thailand.

Despite many of the findings suggesting a causal relationship between temperature and Dengue infections, it cannot be assumed that temperature is the main factor leading to the observed epidemiology. Humidity and precipitation are two climatic factors that can be assumed to strongly influence \emph{Aedes} reproduction, as water sources are crucial for mosquito breeding \citep{Abdullah2022} \citep{Li1985}.  Those factors were not recognized in this analysis, but will possibly provide valuable context to the findings, as Thailands monsoon seasons significantly differ in humidity and precipitation \citep{Kripalani1995}.
