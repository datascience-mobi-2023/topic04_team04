\section{Discussion}
Our results show that increasing temperatures, resulting from climate change as well as recurrent weather phenomenons, are significantly associated with increasing DHF cases.

By a linear model, a general temperature increase over the studied time period can be shown. Nevertheless, it has to be taken into account that the temperature development over the years is not a linear process due to the many influencing factors. Thus, linear models can not be used for exact predictions or identification of patterns. 

It was shown that DHF cases and temperature have a similiar seasonal pattern and general trend: Time periods with high temperatures correspond to periods with high case numbers. Thus, there is indeed a strong relationship between temperature and dengue cases. However, whether there is a causal relationship cannot be concluded from Figures \ref{fig:decomp_ts_dengue} and \ref{fig:Trend_temp_cases}. Nevertheless, the observation that rising temperatures preceded the rise of dengue cases by a few months suggests an influence of high temperatures on \emph{Aedes} biology and virus transmission. 
Higher temperature can indeed positively influence Dengue transmission, as it alters \emph{Aedes} biology: It makes reproduction of the mosquitoes more efficient by shortening the maturation period of  larvae. Virus transmission is promoted as well, as the virus needs less time to spread in its host and make it infectious. These two effects lead to increasing numbers of infectious mosquitoes \citep{Anyamba2019}. The optimal temperature has been observed to lie at 29.3 °C with an small daily range of deviation \citep{Liu2014}. The temperatures during times of high DHF cases lie around this value(... ref zu descriptive Graph?...).

The trends in Figure \ref{fig:Trend_temp_cases} show the second highest DHF case numbers in 2015-16, which is the period of one of the strongest El Niño events since 1950. El Niño increased the temperature in Thailand in that timeframe, leading to higher DHF cases due to effects of temperature on Dengue infections elaborated above \citep{Anyamba2019}, which our analysis confirms. Previous studies have also found this effect by applying an autoregressive model: Similar effects of El Niño on Dengue incidences occured between 1996 and 2005 \citep{Tipayamongkholgul2009}.

Despite these findings suggesting a causal relationship between temperature and Dengue infections, it cannot be assumed that temperature is the main factor leading to the observed epidemiology. Humidity and precipitation are two climatic factors that can be assumed to strongly influence \emph{Aedes} reproduction, as water sources are crucial for mosquito breeding.  Those factors were not recognized in this analysis, but will possibly provide valuable context to the findings, as Thailands monsoon seasons significantly differ in humidity and precipitation. (...Quelle???...)





	