\documentclass[parskip]{scrartcl}
\usepackage{preamble}

% Title Page
% Definitionen für die Titelseite
% Change the content of the second bracket pair to what you need
\newcommand{\workauthor}{Thaddeus Kühn, Dana Levi, Frederik Racky, Fiona Tanner}
\newcommand{\worktitle}{Temperature sensitivity of dengue in Thailand}
\newcommand{\workyear}{2023}
\newcommand{\supervisor}{Dr. Stella Dafka}

\begin{document}

\begin{titlepage}
    % HIER NICHTS ÄNDERN SONDER BEI DEN DEFINITIONEN OBEN DRÜBER    
    {\usekomafont{title}
    {\vspace*{1cm}\Huge \worktitle{}}}
    
    \vspace{\stretch{2}}
    
    {\Large \workauthor{}}

    
    \vspace{\stretch{6}}
    
    	Bioinformatics Project group 4 team 4 climate sensitive infectious diseases Summer Term \workyear{}
    
    Supervisor: \textbf{\supervisor{}}
    
    \vspace{\stretch{4}}
   
    
    \vspace{\stretch{4}}

    
\end{titlepage}




\begin{abstract}

    The full abstract goes here
    

\end{abstract}



\tableofcontents

%    Abkürzungsverzeichnis
{\setlength{\parskip}{0.2cm}
\section*{Abbreviations}
    \begin{acronym}[LC-MS/MS23]
        % A B C D E F G H I J K L M N O P Q R S T U V W X Y Z        
        % Abkürzungen
        \acro{DHF}{Dengue Hemmorhagic Fever}
        \acro{ARIMA}{Autoregressive moving average}
        \acro{GAM}{General additive model}
        \acro{ADF}{Augumented Dickey Fuller test}
       \acro{ACF}{Autocorrelation function}
       \acro{pACF}{Partial autocorrelation function}
       \acro{AIC}{Akaike's information criterion}
        
        % Formelzeichen
        
        
        % als benutzt markierte Acronyme    
        
        
    \end{acronym}
}
\section{Introduction}\label{sec:introduction}

The Dengue virus is a vector borne virus, consisting of four serotypes (DENV 1-4).  It is transmitted to humans by mosquitoes, more specifically by \emph{Aedes aegypti} and \emph{Aedes albopictus} \citep{Phanitchat.2019}. The resulting infection goes asymptomatic in many cases, but can also cause Dengue Hemmorhagic Fever (DHF), characterized by symptoms like fever, headache, joint and muscle pain as well as (internal) bleeding and bruising \citep{Gubler.1998}.
Dengue is an emerging public health issue, with half of the worlds population at risk of infection. 390 million cases per year are estimated worldwide, with most of them not showing symptoms and thus not being reported. Affected areas range from sub-tropical to tropical regions, with south-east Asia, including Thailand being one of the most seriously affected regions. Newly affected areas also include Europe \citep{WHO.2023}.
The disease control of dengue is challenging due to the absence of an effective treatment or vaccine, which leaves only the treatment of symptoms with painkillers like paracetamol \citep{WHO.2023}. Responsible institutions in affected areas currently focus on prevention, vector control, case control and prediction of possible future outbreaks \citep{Phanitchat.2019}. 

An important factor in predicting the epidemiological dynamics of Dengue is the climate: climate fluctuations due to recurring weather phenomenons and climate change are shown to influence \emph{Aedes} biology and infections \citep{Descloux2012, Phanitchat.2019}.
In several studies, maximal temperature has an effect on dengue transmission \citep{Descloux2012}, being associated with higher incidence \citep{Phanitchat.2019} between 27 °C and 29,5 °C. Extreme global climate events like el Niño have been shown to affect disease outbreaks like Dengue as well \citep{Anyamba2019}.

In Thailand, significant climate factors are monsoon (hot and rainy) between may and october and dry season between november and april. It has been shown that in northern districts of Thailand, there has been a detectable rise in temperature since the mid 20th century \citep{Masud2016}.

In this analysis, we are going to investigate the correlation between temperature and Dengue in Thailand from 2006 to 2020 to assess the significance of climate change, recurring climate fluctuations and extremes and geographical factors on Dengue infections. This will allow us to model a forecast for the development of Dengue infections in Thailand in the near future. 

\section{Material and Methods}

\subsection{Material}

	\paragraph{ERA5 data (climate)}
	The ERA5 database is a global climate database by the European Centre for Medium-Range Weather Forecasts (ECMWF), covering the earth in a 31 km horizontal grid up to 80 km in the atmosphere in the time period from 1950 to present. It was generated from measurements of various climate variables combined with a reanalysis of existing data and past reanalysis data to accurately model and complete the dataset in the given resolution \citep{Hersbach2020}. For this project, monthly temperature data 2m above ground for every province in Thailand in the timeframe of 2006 - 2020 was extracted.
	\paragraph{Dengue data}
	The Dengue case numbers are retrieved from annual infectious disease reports published by the Thailand ministry of health. Monthly case numbers of DHF for every province in the timeframe of 2006 - 2020 are used in this project. 
	
	The datasets are used with a resolution at province level. As of 2011, Thailand has a total of 77 provinces, but had 76 provinces before 2011, as Bueng Kan was split from Nong Khai in 2011. For better compatibility of the data before and after 2011, the two new provinces are merged into a province equivalent to Nong Khai before 2011. This excludes data used for mapping, there the temperature and incidence values of Nong Khai are copied and used for Bueng Kan for compatibility with the spatial data. 
	
	\paragraph{Climate forecast}
	For GAM modeling, temperature  data from the CORDEX climate model is used. It includes the temperature 2m above ground for June to August (south west monsoon) of the years 2021 -2040 at a 22 km grid \citep{Copernicus2019}. 
	\paragraph{Population data}
	We used ... population data of every Thailand province from 2006 - 2020 (Quelle?).
	
	\paragraph{Spatial data}
	To associate our data with the different provinces and visualize it, we use spatial data of Thailand’s provinces. The two main data types in spatial data are vector-data and raster-data. Vector-data consists of a list of points with their exact location, which can then form lines or polygons. Raster-data assigns a value to every square of a raster. In this case, maps consisting of polygons for each province are used \citep{sds}. With the sf package, objects associating our data for each province with its coordinates and polygons are created, describing its shape. The function geom\_sf of ggplot2 are used for mapping. 
	
	\subsection{Methods}
	
	\paragraph{Descriptive analysis}
	Linear regression is a method to describe a linear relationship between variables (temperature and time), using the minimum sum of squares between regression line and data points to identify possible trends \citep{Schneider2010}. 
	All missing data points were excluded for the analysis. For summarizing the data, descriptive parameters like mean, standard deviation, maximum and minimum were computed and the distribution of the data was visualized in a histogram. For better comparison, a log transformation was applied to the incidence curve. 
	For visualizing the relationship between incidence and temperature, a smoothing function was used. This function minimizes the sum of squares of the residuals, getting a smooth curve based on the data. Subsequently, the correlation between our two variables was determined via a Pearson correlation.

	
	

	\paragraph{ARIMA}
	
To predict the development of the dengue cases with an ARIMA model, a time series with the total dengue cases of Thailand was created. A time series can be decomposed in the components trend, seasonality and random. 
A requirement for fitting an ARIMA model is a stationary time-series. This is obtained, when the mean value doesn’t change over time, the variance doesn’t increase and the seasonality effect is minimal \citep{Prabhakaran2017}. Two tests were used to test for stationarity of the data. The Augmented Dickey-Fuller (ADF) test examines whether the time series has a unit root, indicating non-stationarity. In the ADF test, the null hypothesis assumes the presence of a unit root, implying non-stationarity, while the alternative hypothesis suggests stationarity. The Kwiatkowski-Phillips-Schmidt-Shin (KPSS) test is also a unit root test but focuses on the presence of a deterministic trend in the series. In contrast to ADF-test, for the KPSS-test the null hypothesis assumes stationarity. If the time series is initially found to be non-stationary, the differences between consecutive observations can be calculated, and the stationarity tests can be applied again \citep{Hyndman2018}.
ARIMA models combine an autoregressive model AR(p) and a moving average model MA(q). 
The autoregressive model computes the current value from previous values and the error term:
	\[y_t = c + \phi_1 y_{t-1} + \phi_2 y_{t-2}+...+ \phi_p y_{t-p}+ \epsilon _t\]
	$\epsilon _t$ = white noise\\
	$1,…,\phi _p$ = parameters\\
	$y_{t-1},…, y_{t-p}$ = lagged values \\
	
	For the moving average the current value consists of the mean value of the time series and weighted current and past error terms: 
	\[y_t=c+\epsilon _t+\theta_1 \epsilon _{t-1}+\theta_2 \epsilon t-2+...+\theta_q \epsilon _t-q\]
	$\theta_1,…, \theta _q $= parameters
	
I(t) is the number of times differencing was performed to make the time series stationary \citep{Venkat2018}.
To find the optimal values for p and q, the Autocorrelation function (ACF) and partial Autocorrelation function (pACF) were evaluated. The ACF plot shows the correlations of a time-series with lags of itself.  The pACF calculates the relationship between a time series and its lag, excluding the influence of linear dependencies among other lags \citep{Prabhakaran2017}.
A second evaluation tool is the auto.arima function. The function automatically fits the best ARIMA model by minimizing the Akaike’s Information Criterion (AIC), which is a criterion for the quality of the model. The auto.arima function can also consider seasonal models. 
Optimal models will yield uncorrelated residuals with zero mean and constant variance. This can be evaluated by plotting the ACF of the residuals and by performing a portmontreau test, for example the Ljung-Box test \citep{Hyndman2018}.

	
	\paragraph{GAM}
	GAM provides insight into the shape and direction of the relationship between temperature and dengue cases. Additionally, it was used to forecast the future dengue case development based on the temperature development prediction. 
	A GAM is a flexible extension of Generalized Linear Models (GLMs) that allows for the modelling of non-linear relationships between the response variable and predictor variables. A linear model can be described as follows:
	$$y=\beta _0+\beta _1 x_1+\beta _2 x_2 +…+\beta _p x_p+\epsilon$$
	GAMs are now a nonparametric form of regression where the linear predictors $(\beta_i x_i)$ of the regression are replaced by smooth functions of the explanatory variables, $f(x_i)$. The model can be defined as:
	$y_i=f(x_i )+\in_i$
	where $y_i$ is the response variable, $x_i$ is the predictor variable, and f is the smooth function \citep{Wood2006}.
	It is called an additive model, because all the $f(x_i)$ functions and therefore their predictor variables contribute individually to the response variable and are added up. The advantage is that the different smooth functions can capture complex relationships by flexibly fitting curves to the data.
	There are two different ways to interpolate the functions of the predictor values. Finding a polynomial with a specific degree that passes through all the data points is a polynomial interpolation approach. Because high-degree polynomials may result into wide oscillation or overfitting, piece-wise interpolation is sometimes more accurate. Here, the data is divided into smaller intervals, each one is described by an individual function. Defining the number of knots determines into how many segments the model is divided. All polynomials together are called splines with a degree of k. They connect the knots one by one. The spline can be differentiated k-1 times. A smaller number of knots makes the response smoother, while a higher k value results in a curve that closely follows the individual data points. Choosing from various types of splines, in the GAM smoothing splines were used, which try to fit the data closely as well as maintaining smoothness \citep{Peri2021}.
	The obtained model includes the relationship between incidence of dengue cases and temperature. Such GAMs can be used to forecast the development of the response variable (incidence of dengue fever) based on the predictor variables (temperature). The predicted values can than be plotted on a map of Thailand. The prediction is based on the average temperature for the months June to August for the south west monsoon over the time period of 2021 until 2040 in Thailand.
	

\section{Results}
	\subsection{Descriptive Analysis}

	\subsection{ARIMA model}
	The time series of total dengue cases in Thailand was additively decomposed in the components trend, seasonal and random, as shown in Figure \ref{fig:decomp_ts_dengue}. The time series shows a seasonal pattern with a frequency of 12 months. The trend shows fluctuation over the time period with two periods of comparatively high dengue cases. 
	
	\begin{figure}[htbp] 
		\centering
		\includegraphics[width=0.7\textwidth]{fig/Decomposition_of_add_ts.png}
		\caption{ Decomposition of the additive time series of the total dengue cases in Thailand}
		\label{fig:decomp_ts_dengue}
	\end{figure}
	
	The composition of the time series of the dengue cases was compared to the decomposition of the time series of temperatures. Both time series have an annual seasonal pattern. The trends show in general similar patterns with differences in the amplitudes and slight shifts on the time axis, as shown in figure y.
	Abb. der trends decomposition
	Bildunterschrift: Figure Y: Trend of the total dengue cases in Thailand compared to the trend of the average temperature in Thailand for the years 2006 till 2020
	
\section{Discussion}
Our results show that increasing temperatures, resulting from climate change as well as recurrent weather phenomenons, are significantly associated with increasing DHF cases.
The relationship between temperature and DHF did not show a significant correlation, but a non-linear pattern was identified. DHF cases exhibited an increase with temperature up to 28°C, followed by a subsequent decrease. Similar temporal patterns were observed, indicating that high temperatures align with high case numbers. By comparing the spatial distribution of temperature and DHF, certain areas with similar patterns were identified. However, it cannot be concluded that provinces experiencing high temperatures also exhibit high case numbers. 
The characterized relationship could be observed in predictions as well. 

By a linear model, a general temperature increase over the studied time period can be shown. Nevertheless, it has to be taken into account that the temperature development over the years is not a linear process due to the many influencing factors. Thus, linear models can not be used for exact predictions or identification of patterns. 

It was shown that DHF cases and temperature have a similiar seasonal pattern and general trend: Time periods with high temperatures correspond to periods with high case numbers. Thus, there is indeed a strong relationship between temperature and dengue cases. However, whether there is a causal relationship cannot be concluded from Figures \ref{fig:decomp_ts_dengue} and \ref{fig:Trend_temp_cases}. Nevertheless, the observation that rising temperatures preceded the rise of dengue cases by a few months suggests an influence of high temperatures on \emph{Aedes} biology and virus transmission. 
Higher temperature can indeed positively influence Dengue transmission, as it alters \emph{Aedes} biology: It makes reproduction of the mosquitoes more efficient by shortening the maturation period of  larvae. Virus transmission is promoted as well, as the virus needs less time to spread in its host and make it infectious. These two effects lead to increasing numbers of infectious mosquitoes \citep{Anyamba2019}.
 
The trends in Figure \ref{fig:Trend_temp_cases} show the second highest DHF case numbers in 2015-16, which is the period of one of the strongest El Niño events since 1950. El Niño increased the temperature in Thailand in that timeframe, leading to higher DHF cases due to effects of temperature on Dengue infections elaborated above \citep{Anyamba2019}, which our analysis confirms. Previous studies have also found this effect by applying an autoregressive model: Similar effects of El Niño on Dengue incidences occured between 1996 and 2005 \citep{Tipayamongkholgul2009}.
Another El Niño event took place in 2018-19, which was another time period with high DHF cases.  In 2019, El Niño led to early monsoon onset \citep{Hu2020}, a possible contributing factor to Dengue transmission. 

 The optimal temperature has been observed to lie at 29.3 °C with an small daily range of deviation \citep{Liu2014}. Another study also proved 29°C to be the temperature of maximal dengue transmission \citep{Phanitchat.2019}. They found 80\% of dengue cases occurring at a mean temperature of between 27.0 and 29.5 °C. Our analysis proved 61\% of infections to occur between 27 and 29.5 °C, aligning with both sources (see Figure \ref{fig:smooth_spline}).
 
In the maps showing mean DHF cases in the different provinces, the northwestern provinces have comparatively high incidences, although the temperatures are lower (see Figure \ref{fig: }). A possible cause for this unexpected observation could be frequent floods during south-west monsoon, associated with high incidences in that region (e.g. Mae Hong Son) \citep{Chaithong2022}. The connection of floods and incidences could result from increased water sources for mosquito breeding grounds and loss of preventive actions and healthcare resources. 

To evaluate the further development of DHF cases in Thailand in the future, two different models were applied for forecasting. The ARIMA model projected a seasonal pattern with a slight upward trend in Thailand from 2021 to 2024, capturing the fundamental characteristics of case development in the country. However, a comparison of the model's predictions for 2017 to 2020 with the reported cases during that period revealed discrepancies in the amplitudes of the model. Since the model only relies on past cases and does not consider nonlinear trends, it cannot accurately predict complex relationships influenced by various factors. Nevertheless, the actual cases still lie in the deviation range, proving the general validity of the model.

Secondly, the GAM was used to predict future DHF cases based on the temperature. The edf value is with 13.82 very close to k with 16, which indicates that the model is using most of the available flexibility to capture the underlying relationship. The edf value is far from 1, therefore a linear correlation can be eliminated. It suggests a wigglier spline as well. It was shown that the smooth function of the temperature has a highly significant impact on the predicting incidence, but the model only explains 4.7\% of the variance in the data. Thus, it can be said that temperature is one significant of many parameters which will influence the development of dengue.
In the future, the east of Thailand will be the main origin of dengue fever outbreaks. This could be very critical, because of the low coverage of health care establishments \citep{Witthayapipopsakul2019}. Although the east will be a hotspot, the model suggests that the average incidence in Thailand will increase compared to the period of 2006 to 2020. 

High fluctuation in the incidence values (sd = 15.7) can’t be seen in temperature values (sd = 2.1 °C). The difference in the standard deviation is even higher in the south monsoon season with a sd for the incidence of 21,6 and a sd for the temperature of 1.2 °C. At this time of the year, other parameters like precipitation have a higher influence on dengue infection. In south east asia the temperature doesn’t fluctuate much over the year (Quelle).
Regional patterns are probably more outbreak specific and may be influenced by extreme weather events. The sd of the average temperature per region is 1.1 °C, which indicates that temperature values are relatively stable over Thailand.

Despite many of the findings suggesting a causal relationship between temperature and Dengue infections, it cannot be assumed that temperature is the main factor leading to the observed epidemiology. Humidity and precipitation are two climatic factors that can be assumed to strongly influence \emph{Aedes} reproduction, as water sources are crucial for mosquito breeding \citep{Abdullah2022} \citep{Li1985}.  Those factors were not recognized in this analysis, but will possibly provide valuable context to the findings, as Thailands monsoon seasons significantly differ in humidity and precipitation \citep{Kripalani1995}.



\bibliography{bib/essay.bib}

\newpage
\section{Appendix}


\begin{figure}[htbp] 
	\centering
	\includegraphics[width=0.7\textwidth]{fig/dist_incidence.png}
	\includegraphics[width=0.7\textwidth]{fig/dist_temp.png}
	\caption{Distribution of incidence (top)
		and distribution of temperature: blue = density, green = normal distribution (bottom)}
	\label{fig:dist_temp_inc}
\end{figure}


\begin{figure}[htbp] 
	\centering
	\includegraphics[width=0.7\textwidth]{fig/temp_over_years.png}
	\caption{Monthly mean temperature of Thailand over the analyzed timeframe (black), linear regression of temperature (red).}
	\label{fig:temp_over_years}
\end{figure}

\begin{figure}[htbp] 
	\centering
	\includegraphics[width=0.7\textwidth]{fig/mean_median_temp.png}
	\caption{Temperature mean, median and difference between mean and median (Thailand average)}
	\label{fig:mean_median_temp}
\end{figure}


\begin{figure}[htbp] 
	\centering


		
		
	\caption{}
	\label{fig:inc_temp_maps}
\end{figure}




	\begin{figure}
		
		\centering
		\begin{subfigure}{0.5\textwidth}
	\includegraphics[width=\textwidth]{fig/Average_temperature.png}
			\caption{Average temperature 2006 - 2020}
			\label{fig:av_temp_map}
		\end{subfigure}
		\hfill
		\begin{subfigure}{0.5\textwidth}
			\includegraphics[width=\textwidth]{fig/Average_incidence_log.png}
			\caption{Average incidence 2006 - 2020\\ (log-transformed)}
			\label{fig:av_inc_log}
		\end{subfigure}
		\hfill
		\begin{subfigure}{0.6\textwidth}
			\includegraphics[width=\textwidth]{fig/Average_incidence.png}
			\caption{Average incidence 2006 - 2020}
			\label{fig:av_inc_map}
		\end{subfigure}
			\begin{subfigure}{0.4\textwidth}
		\includegraphics[width=\textwidth]{fig/GAM_map.png}
			\caption{GAM forecast: average incidence for 2021 - 2024}
			\label{fig:GAM_map}
		\end{subfigure}
		
		\caption{Comparison of distributions of temperature to incidence and GAM forecast}
		\label{fig:maps}
		
	\end{figure}
	
\end{document}

